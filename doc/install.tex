\Chapter{Installing XGAP}

Installing {\XGAP} should be easy once you have installed gap itself. We
assume here that you want to install {\XGAP} in its standard location,
which is in the ``pkg'' subdirectory of the main {\GAP4} installation.

\Section{Overview}

You have to perform the following steps to install {\XGAP}:

\beginlist
\item{--} Get the sources
\item{--} Unpack the sources with the <unzoo> utility
\item{--} Use the <configure> script to adjust everything to your specific
  system
\item{--} Compile the C part of {\XGAP}
\item{--} Edit a certain startup script and install it in an executable
      location in your system
\endlist

\Section{What you need to install XGAP}

Being a graphical user interface to {\GAP}, {\XGAP} of course needs
graphics. At the moment this means that you need the X window system in the 
Version 11 Release 5 or newer. So you *cannot* use {\XGAP} on a Macintosh
or a Microsoft Windows computer. On the other hand the type of Unix you use 
should not matter. Please contact the author or post to the gap-trouble
mailing list, if you encounter problems with certain system configurations.

Because {\XGAP} consists of a C-part you need a C compiler.

\Section{Getting and unpacking the sources}

You can download the sources from the same places as {\GAP}. So the main
FTP servers are:

\begintt
ftp://ftp-gap.dcs.st-and.ac.uk/pub/gap/gap4/
ftp://ftp.math.rwth-aachen.de/pub/gap4/
ftp://ftp.ccs.neu.edu/pub/mirrors/ftp-gap.dcs.st-and.ac.uk/pub/gap/gap4/
ftp://pell.anu.edu.au/pub/algebra/gap4/
\endtt

You need only one file with the name ``xgap4.zoo''
%FIXME: Filename
which is in the subdirectory for the share packages.
%FIXME: Is that so?
When you installed {\GAP} you used the utility <unzoo> to unpack the
distribution. You will need this here again. See the {\GAP}-manual for
instructions on how to get and compile this.
%FIXME: Reference
You now change your current directory to the location where you installed
{\GAP} (you typed an <unzoo>-command, then a new directory called
``gap4'' or something like that was created, this directory is the right
place). The standard location would be:

\begintt
# cd /usr/local/lib/gap4
\endtt

Now you extract the sources for the {\XGAP} share package:

\begintt
# unzoo -x xgap4.zoo
...
\endtt
%FIXME: Place some output here, FILENAME?

The <unzoo> utility unpacks the files and stores them into the apropriate
subdirectories. {\XGAP} resides completely in the following subdirectory
(assuming standard location):

\begintt
/usr/local/lib/gap4/pkg/xgap
\endtt

\Section{Configuring and Compiling the C part}

You have to change your current working directory to the ``xgap''
subdirectory. You do this by

\begintt
# cd pkg/xgap
\endtt

if your current working directory is the one, where you used
<unzoo>. There you invoke the <configure> script by:

\begintt
# ./configure
creating cache ./config.cache
checking for make... make
checking host system type... i686-unknown-linux2.0.34
checking target system type... i686-unknown-linux2.0.34
checking build system type... i686-unknown-linux2.0.34
checking for gcc... gcc
checking whether the C compiler (gcc  ) works... yes
checking whether the C compiler (gcc  ) is a cross-compiler... no
checking whether we are using GNU C... yes
...
updating cache ./config.cache
creating ./config.status
creating Makefile
\endtt

$\ldots$ indicate omissions. 
This script tries to determine, which kind of operating system and
libraries you have installed and configures the source
accordingly. Normally this should produce some output but no error
messages. The last step of the script produces some makefiles which are
used to compile the code. You do this by typing

\begintt
# make
if test ! -d bin;  then mkdir bin;  fi
if test ! -d bin/i686-unknown-linux2.0.34-gcc;  then mkdir bin/i686-unknown-linux2.0.34-gcc;  fi
cp cnf/configure.out bin/i686-unknown-linux2.0.34-gcc/configure
( cd bin/i686-unknown-linux2.0.34-gcc ; CC=gcc ./configure --target=i686-unknown-linux2.0.34  )
creating cache ./config.cache
checking for gcc... gcc
checking whether the C compiler (gcc  ) works... yes
checking whether the C compiler (gcc  ) is a cross-compiler... no
checking whether we are using GNU C... yes
...
creating ./config.status
creating Makefile
creating config.h
make[1]: Entering directory `/usr/local/lib/gap4/pkg/xgap/bin/i686-unknown-linux2.0.34-gcc'
gcc -I. -g -O2 -I/usr/X11R6/include  -o xcmds.o -c ../../src.x11/xcmds.c
gcc -I. -g -O2 -I/usr/X11R6/include  -o utils.o -c ../../src.x11/utils.c
...
make[1]: Leaving directory `/usr/local/lib/gap4/pkg/xgap/bin/i686-unknown-linux2.0.34-gcc'
\endtt

Now all C sources are compiled and a binary executable is built. It is
stored in a subdirectory of the ``bin'' subdirectory in your ``xgap''
directory. The name of this location has something to do with your
installation. It could for example be

\begintt
bin/i686-unknown-linux2.0.34-gcc/xgap
\endtt

if you compile on a Linux system using the GNU-C-Compiler.

\Section{Installing the Startup Script}

To make the startup of {\XGAP} more convenient there is a startup script
which contains also some configuration information like the position of
your {\GAP} installation. It is in the ``bin'' directory within the
``xgap'' tree and is called ``xgap.sh''. You should edit it until the line

\begintt
##  STOP EDITING HERE !!!!!!!!!!!!!!!!!!!!!!!!!!!!!!!!!!!!!!!!!!!!!!!!!!!
\endtt

You have to specify the directory where {\GAP} is installed (``GAP\_DIR''), 
the amount of memory that {\GAP} should use as initial workspace
(``GAP\_MEM''), the name of the {\GAP}-executable (``GAP\_PRG'') and the
name of the {\XGAP}-executable (``XGAP\_PRG''). The first three are exactly 
the same things that you had to edit in the main {\GAP} startup script.
After that you have the possibility to control the behaviour of the {\XGAP}
startup script. You can specify whether {\XGAP} goes into the background
(``DAEMON'') and whether it prints out information about its parameters
(``VERBOSE''). Note that it is possible to combine ``DAEMON=YES''
and ``VERBOSE=YES'' because the script actually runs in the foreground and
only the C program is put into the background.

Copy the modified script to a location that people have in their ``PATH''
environment variable, for example to ``/usr/local/bin''.

This completes the installation.

Perhaps the easiest way is to leave the names of the executables as they
are in the script and make a symbolic link to the directory with the system 
type with the name of the machine you want to run {\XGAP}:

\begintt
# cd /usr/local/lib/gap4/pkg/xgap/bin
# ln -s i686-unknown-linux2.0.34-gcc beeblebrox
\endtt

if ``beeblebrox'' is the hostname of the machine you want to run {\XGAP}
on.

\Section{Installing in a different than the standard location}

It could happen that you do not want to install {\XGAP} in its standard
location, perhaps because you do not want to bother your system
administrator and have no access to the {\GAP} directory. In this case you
can unpack {\XGAP} in any other location with the <unzoo> command as
described above. Let us call this directory ``{\XGAP} directory'' for the
moment. You get a ``pkg'' directory and in there a ``xgap''
subdirectory with all the files of {\XGAP} in it. You follow the
standard procedure with two exceptions:

Before you can configure and compile {\XGAP} you need a symbolic link
``cnf'' which is in the {\XGAP} directory and points to the ``cnf''
directory in the main {\GAP} directory.

When editing the startup script you enter the name of the {\XGAP} directory 
for the variable ``XGAP_DIR''. The script will automatically launch {\GAP}
with two directories as library path such that all {\GAP} and {\XGAP}
libraries will be found.

%%% Local Variables: 
%%% mode: latex
%%% TeX-master: "manual"
%%% End: 
