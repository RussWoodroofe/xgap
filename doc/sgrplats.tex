\Chapter{Subgroup Lattices}

{\XGAP} provides a graphical interface to the  lattice or partial lattice
of subgroups of groups.  For finitely  presented groups it gives you easy
access to the   low  index,  prime quotient   and   Reidemeister-Schreier
algorithms in order to build a partial  lattice interactively.  For other
types of groups  it provides easy access to  many of  the group functions
(for example, the normalizer, normal subgroups, and Sylow subgroups).

This chapter explains how to use this interface.  Using various examples
the first sections "The Subgroup Lattice of the dihedral group of order 8",
"A Partial Subgroup Lattice of the symmetric group on 6 points", "A Partial
Subgroup Lattice of the Cavicchioli Group", and "A Partial Subgroup Lattice
of the Trefoil Knot Group" explain most features.  The later sections
give details about the various options and menus available.  This chapter
will not describe how to write your own programs using the graphic
extensions supplied by {\XGAP}, see chapters "Graphic Sheets - Basic graphic
operations" to "Graphic Posets" for details.

It is assumed that you have already started {\XGAP}.  On most systems you
do this by typing

\begintt 
user@host:~> xgap 
\endtt

on  the  command line.   Ask your  system administrator  if this does not
work.  This   command  will create   a new window,   the so called {\GAP}
window, in which {\GAP} is awaiting your input.   Depending on the window
system and window manager  you use, placing a new  window on your  screen
might be  done  automatically or might  require  you to use the  mouse to
choose a position for the  window and pressing the  left mouse button  to
place the window.

The small arrow   or cross you see on   your screen is  called a pointer.
Although the device used to move this pointer can be anything, a mouse, a
track ball, a glide-pad, or even something as exotic as a rat, we will use
the term mouse to refer to this pointer device.

In  case that some computation  takes  longer than expected, for instance
the low index and the prime quotient can be quite time consuming, you can
always interrupt  a computation  by  making the {\GAP}  window active and
pressing <CTR-C>.  Again, making  a  window active  is system and  window
manager dependent.   In most cases  you either have  to  move the pointer
inside  the {\GAP} window or you  have to click on the   title bar of the
{\GAP} window.

%%%%%%%%%%%%%%%%%%%%%%%%%%%%%%%%%%%%%%%%%%%%%%%%%%%%%%%%%%%%%%%%%%%%%%%%%
\Section{The Subgroup Lattice of the Dihedral Group of Order 8}

This  section   gives  you an   example    on how  to    use the function
`GraphicSubgroupLattice' (see "GraphicSubgroupLattice" for  details), which 
will  display the Hasse diagram of the subgroup lattice of a given group.

Using the dihedral group of size $8$ as example the following will show you
most features of the `GraphicSubgroupLattice' program.  This exercise is
best carried out in front of {\XGAP}, trying the various commands yourself.

First  you   have to define a  group   in {\GAP},  this example  uses the
dihedral group defined as permutation group.

\begintt
gap> d8 := DihedralGroup(8);
<pc group of size 8 with 3 generators>
gap> SetName(d8,"d8");
\endtt

Now you ask for a graphical display by

\begintt
gap> s := GraphicSubgroupLattice(d8);
<graphic subgroup lattice "GraphicSubgroupLattice">
\endtt

{\XGAP} will open a window containing a new graphic sheet, a menu bar
(menus are described below) above the graphic sheet and a title.  On most
systems the title will be either below the graphic sheet or above the menu
bar.  The dimension of the graphic sheet is fixed, changing the size of the
window will *not* change the size of the graphic sheet, see
"GraphicSubgroupLattice, Poset Menu" how to resize the graphic sheet.  It
is possible that the graphic sheet is larger (depending on the lattice it
might be much larger) than the window.  In this case the window will
contain so called scrollbars which allow you to select the portion of the
graphic sheet which will be displayed. 

{\XGAP} first shows only the whole group (which is already selected) and
the trivial subgroup, connected by a line indicating inclusion. 

`ConjugacyClassesSubgroups' computes and returns the conjugacy classes of
subgroups, so summing  up the sizes  of the classes   tells you how  many
elements the lattice has.

\begintt
    gap> Sum( List( ConjugacyClassesSubgroups(d8), Size ) );
    10 
\endtt

$10$ is small enough to use `AllSubgroups' without painting the screen
black.  After you have clicked this menu entry in the `Subgroups' menu you
see the complete Hasse diagram in the graphic sheet.

The   following initial  remarks  can    be  made  about  the   graphical
representation of the subgroup lattice:

\beginlist
\item{--} The vertex representing the trivial subgroup is labeled $1$.
  
\item{--} Vertices representing subgroups of the same size are drawn at the
  same height. They are said to be ``on the same level''.  In our example
  the subgroups that belong to the vertices $2$, $4$, $5$, $8$ and $9$ all
  have size 2, and the subgroups of $3$, $6$, and $7$ have size $4$.  The
  default behaviour is to place a vertex above another one if the size of
  the subgroup represented by the first vertex is larger than the size of
  subgroup of the second, but see `GraphicSubgroupLattice' for details. At
  the right edge of the graphic sheet each level is labeled with the index
  of the subgroups contained.
  
\item{--} Vertices belonging to the same conjugacy class are placed closely
  together.  In our example the subgroups of $4$ and $5$ form one conjugacy
  class.
\endlist

The initial placement of the vertices chosen by `GraphicSubgroupLattice'
might not be optimal or you might want to choose a different one in order
to exploit certain features of the diagram.  It is therefore possible to
move the vertices around using the mouse.

The mouse together with the left mouse button can be used to move and
select vertices. A selected vertex is represented by a thicker circle,
colored red if your screen supports color.  For example, in order to *move*
vertex $4$ use the mouse to place the pointer inside the circle around $4$
and press the *left* mouse button.  Keep the mouse button pressed and start
moving the mouse.  The vertex will now follow the pointer.  Because of the
height restrictions given by the size it is not possible to move $4$ above
$6$ or below $1$. It must always stay within its level. If you release the
left mouse button vertex $4$ will stay at its current position and the rest
of the conjugacy class (in this example $5$) will be moved to this new
position.

In order to *select* vertex $G$ place the pointer inside the circle around
$G$, press the *left* mouse button and release it immediately.  Do not
move the mouse while you hold down the left mouse button.  Vertex $G$ now
has a slightly thicker boundary and is red if you have a color screen.
There are two different ways to select more than one vertex, see "A Partial
Subgroup Lattice of the Symmetric Group on 6 Points" or
"GraphicSubgroupLattice, Selecting Vertices".

On the top of the window, above the graphic sheet,  you can see a list of
menu names: ``GAP'', ``Poset'', and ``Subgroups''.  In order to
open any of these *pull  down menus* place the  pointer inside the button
containing   the menu name  and press  the  left  mouse button.  Keep the
button pressed.  A pull down menu will be shown and by moving the pointer
down you  can choose  a   menu entry.  By choosing    an entry and   then
releasing  the  mouse button  the entry   is  selected, the corresponding
function is executed and  the pull down menu is  closed.  If  you release
the mouse button while the pointer is outside the pull down menu the menu
is closed without selecting any entry.

Now select ``Change Labels'' from the ``Poset'' menu.  If this entry is not
available you have failed to select vertex $G$.  After selecting ``Change
Labels'' a small dialog box is opened asking for a label.  Type in `D8' and
press the <return> key or click on <OK>.  The label of vertex $10$ will now
be changed to ``$D8$''. Note that in the X Window System you have to move
the pointer on the text field if you want to edit the label.

In order to find  out which vertex  represents the centre of $D_8$, first
select vertex $D8$ and then the menu entry ``Centres'' from the ``Subgroups''
menu.   In  case of  a  color screen, vertex $D8$   will  be selected and
colored  red, and vertex  $2$ will  be selected  and  colored green.  The
color green  indicates that vertex $2$  is  the result  of a computation.
There will also be a message in the {\GAP} window  saying that vertex $2$
represents the centre of the group belonging to vertex $D8$.

\begintt
#I  Centres (D8) --> (2)
\endtt

Most of  the menu entries  in ``Subgroups'' should be self-explanatory, for
details and  the   difference   between  ``Closure''  and    ``Closures'' see
"GraphicSubgroupLattice, Subgroups Menu".

If you have selected some vertices (in the example $D8$ and $2$ are now
selected), and you want to investigate the subgroups corresponding to these
vertices further in {\GAP}, the function `SelectedGroups' will return a
list of these subgroups:

\begintt
gap> SelectedGroups(s);
[ D8, Group([ f3 ]) ]
\endtt

On the other hand, the functions supplied via the ``Subgroups'' menu are by
far not all functions applicable to groups.  In order to show results of a
computation in {\GAP} in the diagram, you can use `SelectGroups'.  The
function `SelectGroups' allows you to mark any set of subgroups of $D_8$ in
the diagram.

For instance,  you can compute the lower central series of this (nilpotent) 
in {\GAP}.

\begintt
gap> l := LowerCentralSeries(d8);
[ D8, Group([ f3 ]), Group([ <identity> of ... ]) ]
gap> SelectGroups(s,l);
\endtt

This lower central series corresponds to the vertices $D8$, $2$ and $1$
which will now be selected.  To summarize the above: the function
`SelectedGroups' can be used to transfer information from the diagram to
{\GAP}, the function `SelectGroups' can be used to transfer information
from {\GAP} to the diagram.

% obsolete:
%While pull down menus are bound to a menu button and  will be pulled down
%as soon as  this button is  pressed, there is  another type of  menu, the
%so called *pop up menu*.  One of these pop up menus can be used to adjust
%the size of the graphic sheet.  In order to adjust the left margin, place
%the pointer  outside of any  vertex and press  the *right*  mouse button,
%keep the button pressed.  A small menu will pop up.

%|    Left Margin
%    Right Margin
%    Top Margin
%    Bottom Margin |

%Select <Left Margin>.   This will narrow  the graphic sheet,  setting the
%new left margin  to the x coordinate of   your mouse click.   Repeat this
%process with the right margin.  Depending on the window system and window
%manager   narrowing  the graphic   sheet   might also   narrow the window
%containing  the  graphic sheet.  If  not,  you  have to use  the standard
%method of your window system/manager to change the size of this window.

In order to finish this example, close the window by selecting ``close
graphic sheet'' from the ``GAP'' menu.  This will close the window
containing the Hasse diagram of $D_8$.

In this example you have learned, how to display the Hasse diagram of the
subgroup lattice of a group using `GraphicSubgroupLattice', how to use the
mouse to move and select vertices, how to select a menu entry and how to
transfer information between the Hasse diagram and {\GAP} using `SelectGroups'
and `SelectedGroups'.

In order to learn more about the menus ``GAP'' and ``Poset'', which were only
mentioned very briefly, see "GraphicSubgroupLattice, GAP Menu", and
"GraphicSubgroupLattice, Poset Menu".  

% obsolete:
%This example also did not discuss
%two additional menus, namely the <Information> and <Sheet Menu>.  The first
%is discussed in "A Partial Subgroup Lattice of the Symmetric Group on 6
%Points" and "GraphicSubgroupLattice, Information Menu", the latter in
%"GraphicSubgroupLattice, Sheet Menu".

%%%%%%%%%%%%%%%%%%%%%%%%%%%%%%%%%%%%%%%%%%%%%%%%%%%%%%%%%%%%%%%%%%%%%%%%%
\Section{A Partial Subgroup Lattice of the Symmetric Group on 6 Points}

This section investigates the subgroup lattice of $S_6$.

\begintt
gap> s6 := SymmetricGroup(6);
Sym( [ 1 .. 6 ] )
gap> SetName(s6,"S6");
gap> cc := ConjugacyClassesSubgroups(s6);;
gap> Sum(List(cc,Size));
1455
\endtt

As there are $1455$ subgroups,  displaying  the whole lattice of  subgroups
would not  be helpful because  there are simply  too many.   Therefore this
example builds only a partial  subgroup lattice.   We  assume that you  are
familiar with  the  general ideas,  mouse   actions and  menus, which  were
discussed in "The Subgroup Lattice of the Dihedral Group of Order 8".

We again start to build a partial lattice, by using
`GraphicSubgroupLattice' (see "GraphicSubgroupLattice").  After you have
entered

\begintt
gap> s := GraphicSubgroupLattice(s6);
<graphic subgroup lattice "GraphicSubgroupLattice of S6">
\endtt

{\XGAP}  will open  a window  containing  a new  graphic  sheet with  two
connected vertices   labeled  $1$ and $G$.     Vertex  $1$ represents the
trivial subgroup and  vertex $G$ the group $S_6$.   Vertex $2$ is already
selected, so it will be red if your screen supports color.

In order  to  find all subgroups   of size $60$, we  cannot   not use the
``Subgroups'' menu directly, so go back into the  {\GAP} window and extract
the conjugacy classes of <cc> whose representatives have size $60$.

\begintt
gap> c60 := Filtered(cc,x->Size(Representative(x))=60);;
gap> s60 := List(c60,Representative);
[ Group([ (1,2)(3,4), (1,3,5) ]), Group([ (1,2)(3,4), (1,2,3)(4,5,6) ]) ]
\endtt

We now use the  function `InsertVertex'  to  add  these two subgroups  to your
partial lattice.  The Hasse diagram now contains four vertices.

\begintt
gap> for g in s60 do InsertVertex(s,g); od;
\endtt

The new vertices are selected automatically.  Selecting ``Conjugate
Subgroups'' from the ``Subgroups'' menu adds the complete conjugacy classes.
In order to find out what type of subgroups we  are looking at, use another
kind of menu not  discussed so far,  namely the <Information>  menu.  Place
the pointer inside  vertex $3$, press the  *right* mouse button and release
it immediately.  This  will  pop  up a new   window,  containing some  text
describing vertex $3$  (as mentioned above,  depending on the window system
and  window manager, placing this  window on the  screen might require some
interaction with the mouse).

\begintt
Size            60
Index           12
IsAbelian       unknown
IsCentral       unknown
IsCyclic        unknown
IsNilpotent     false
IsNormal        unknown
IsPerfect       true
IsSimple        unknown
IsSolvable      false
Isomorphism     unknown
\endtt

Place the pointer on top of the entry  ``Isomorphism'' and press the *left*
mouse button.  After a while this entry is changed to

\begintt
Isomorphism     [ 60, 5 ]
\endtt

telling you that the subgroup represented by vertex $3$ is isomorphic to
the alternating group on five symbols.  The notation `[ 60, 5 ]' comes out
of the small groups library and is the only information about the
isomophism type we can get from {\GAP4} up to now. Select <close> to close
the <Information> menu.  Repeat this with vertex $2$, you will see that the
subgroup of vertex $2$ is also isomorphic to $A_5$, however these two $A_5$
inside $S_6$ are not conjugate in $S_6$.  The information menu is described
in detail in "GraphicSubgroupLattice, Information Menu".

Now we want to compute the  normalizers of the  elements of the conjugacy
class containing the  subgroup of vertex  $3$.   You could either  select
vertex  $3$ and  then  <Normalisers>  and repeat   this process for   the
vertices $9$ to $13$, or you can first select the vertices $3$, $9$ to $13$
and  then select <Normalizers>.  But how  to select more than one vertex?
If you first select $3$ and  then $9$, vertex $3$  will get deselected as
soon as $9$ gets selected.  However, if you  select vertex $3$, place the
pointer inside vertex $9$, hold down the <SHIFT> key on your keyboard and
then select vertex $9$ using  the left mouse button,  vertex $9$ will  be
selected in addition to  vertex $3$.  Another  method to select more than
one  vertex is  to  use  the  rubber   band  to catch vertices  inside  a
rectangle.  Place  the pointer  left and  a  bit  higher than  vertex $3$
*outside*  any other vertex.   Press the *left*  mouse button and hold it
down.  Now, using  the mouse, move the  pointer right and slightly  below
vertex $13$.  You  see a rectangle, one corner  at your start position and
the other   following the pointer.   If  vertex $3$,  $9$ to $13$  are all
inside this rectangle, release the mouse  button.  Now these vertices are
selected.  Select <Normalizers> from the  <Subgroups> menu to compute and
display the normalizers.

Now select vertex $3$ and $8$ and compute the intersection.  The
intersection is of size $10$.  Select this intersection and use
`SelectedGroups' to get a {\GAP} record describing the subgroup.

\begintt
gap> l := SelectedGroups(s);
[ Group([ (2,3)(4,6), (1,2)(3,4) ]) ]
gap> u := l[1];
Group([ (2,3)(4,6), (1,2)(3,4) ])
\endtt

In order  to find out which subgroups  of the complete lattice  lie above
the subgroup <u> you can  use ``Intermediate Subgroups''. You select the
whole group in addition to <u> and choose ``Intermediate Subgroups'' in the 
``Subgroups'' menu. You get 7 groups, some of them are already in the
lattice, the others are added.

There is another feature we have not seen yet.  Close the current graphic
sheet and start again with a fresh one.

\begintt
gap> Close(s);
gap> s := GraphicSubgroupLattice(s6);
<graphic subgroup lattice "GraphicSubgroupLattice of S6">
\endtt

In order to compute a Sylow $2$ subgroup select ``Sylow Subgroup'' from the
``Subgroups'' menu.   A small dialog box  will  pop up asking  for a prime,
type in  $2$ and press <return> or  click on <OK>.   Now select  this new
vertex $2$   representing the Sylow $2$  subgroup  and compute its normal
subgroups.  This is rather slow because the  function checks for each new
vertex if the corresponding  subgroup is conjugate to  an old one  of the
same size.  


%obsolete?:
%There  is however a  limit on the  index of the normalizer or
%subgroup up to which this test is performed.
%
%\begintt
%gap> s.limits.conjugates;
%100 
%\endtt
%
%If  `limit.conjugates' contains a   positive  number  the index  of   the
%normalizer of the new subgroup is checked. If `limit.conjugates' contains
%a negative number, only the index of the  subgroup itself is checked.  If
%the index  is less  than the  absolute  value then the conjugacy  test is
%carried out.  Change this limit to $-10$.
%
%\begintt
%gap> s.limits.conjugates := -10;
%-10 
%\endtt
%
%Select vertex $3$   and compute a  random conjugate  by selecting <Random
%Conjugate> from the <Subgroups> menu.  Select this new vertex and compute
%its  normal subgroups.  You will  see  that now  the vertices  of the new
%subgroups appear  faster on  the   graphic sheet, however,  they  have  a
%different shape.  They are  represented by a square  instead of a circle.
%A square  is meant as  a danger sign,  because {\GAP} has not  checked if
%subgroups associated  with square vertices  are  conjugate to any  of the
%other subgroups.  Therefore these subgroups might be (and in this example
%will be) conjugate to an old subgroup.

%FIXME:  do we want this feature?
In order  to untangle the Hasse diagram you  can  move the new (selected)
vertices  en block.  Hold down  the <SHIFT> key and   move any of the new
vertices to the right.  All selected vertices will automatically be moved
as soon as you release the left mouse button.

This is  now the  end  of the investigation   of  the (partial)  subgroup
lattice   of  $S_6$, close the    graphic  sheet(s) using  ``close graphic
lattice'' of the ``GAP'' menu.


%%%%%%%%%%%%%%%%%%%%%%%%%%%%%%%%%%%%%%%%%%%%%%%%%%%%%%%%%%%%%%%%%%%%%%%%%
\Section{A Partial Subgroup Lattice of the Cavicchioli Group}

This  section  investigates the  following  finitely presented group, the
so called Cavicchioli-group $C_2$.
$$
\langle a, b \;;\; aba^{-2}ba=b, (b^{-1}a^3b^{-1}a^{-3})^2=a^{-1}\rangle.
$$

The following {\GAP} commands define $C_2$.

\begintt
gap> f := FreeGroup( "a", "b" );  a := f.1;;  b := f.2;;
Group( a, b )
gap> c2 := f / [ a*b*a^-2*b*a/b, (b^-1*a^3*b^-1*a^-3)^2*a ];
Group( a, b )
gap> c2.name := "c2";; 
\endtt

We again assume that you are familiar with the general ideas, mouse actions
and menus, which were discussed in "The Subgroup Lattice of the Dihedral
Group of Order 8" and "A Partial Subgroup Lattice of the Symmetric Group on
6 Points".

In order  to build a  partial lattice of a  finitely presented group, you
again use the function  `GraphicSubgroup\-Lattice'.  But if the first  argument
to `GraphicSubgroupLattice' is a finitely presented group the available menus
are different  from the example in  the previous section.  After you have
entered

\begintt
gap> s := GraphicSubgroupLattice(c2);
<interactive graphic lattice> 
\endtt

{\XGAP} will open a window  containing a new  graphic sheet.  Compared to
the  interactive  lattice of  a permutation   group  as described  in the
previous section, there are the following differences:

-- There is only one  vertex instead of two.   This vertex labeled $1$ is
not the trivial  group, as in the last  two sections but  the whole group
$C_2$.  There is no vertex for the trivial subgroup (yet).

-- The shape of  vertex $1$ is not a  circle but a diamond.  Whenever you
see a diamond this  means that the subgroup  belonging to this vertex  is
normal in the whole group *and* {\GAP} knows this fact.

-- If you pull down the <Subgroups> menu, you will see  that this menu is
now   very different.   It gives  you   access to various algorithms  for
finitely presented  groups  but most of  the  entries  from the last  two
examples  are missing because most of  the {\GAP}  functions behind these
entries are not applicable to (infinite) finitely presented groups.

This  example will show  you how to prove that  $C_2$ is infinite.  First
look at the abelian invariants in order to see what the commutator factor
group is.   In  order  to  compute the abelian    invariants pop  up  the
<Information> menu.   This is done in exactly  the same manner  as in the
previous section.  Place the pointer inside vertex $1$, press the *right*
mouse button and release   it  immediately.  This <Information> menu   is
described  in   detail in  "GraphicSubgroupLattice  for FpGroups, Information
Menu".

\begintt
Index                  1
IsNormal               true
Abelian Invariants     unknown
Presentation           2 gens, 2 rels
Coset Table            known
Factor Fp Group        unknown 
\endtt

This tells you what {\GAP} already knows  about the group associated with
vertex $1$.   In order to compute the  abelian invariants click onto this
line.  After a while this entry will change to

\begintt
Abelian Invariants     perfect 
\endtt

telling you  that $C_2$  is perfect.   So none   of the  <Subgroups> menu
entries <Abelian  Prime Quotient>,  <All Overgroups>,  <Conjugacy Class>,
<Cores>,  <Derived Subgroups>,  <Intersection>,   <Normalizer> or  <Prime
Quotient> will compute any new subgroups.

In order  to avoid accidents  the menu entries  <Abelian Prime Quotient>,
<All Overgroup>, <Conjugacy Class>, <Epimorphisms>, <Low Index Subgroups>
and  <Prime  Quotient> from the  <Subgroups>  menu are only selectable if
exactly one vertex is selected because the functions behind these entries
are in general quite time and space consuming.

Close the <Information> window and select <Low  Index Subgroups> from the
<Subgroups> menu.  A small dialog box will  pop up asking  for a limit on
the index.  Type in $12$ and press <return> or click on <OK>.  In general
it  is hard to say what   kind of index  limit  will still work, for some
groups even $5$ might be too  much while for others  $20$ works fine, see
also "LowIndexSubgroupsFpGroup".

{\GAP} computes $10$ subgroups  of index $11$  and $8$ subgroups of index
$12$.  If you now start to check the abelian invariants of the index $12$
subgroups you will  find out that all  subgroups represented by  vertices
$2$ to $9$  have  a finite  commutator  factor group except  the subgroup
belonging  to  vertex  $3$    which has  an infinite    abelian quotient.
Therefore the group $C_2$ itself is infinite.

Now we want to investigate $C_2$ a little further using {\GAP}.  Selected
vertices  $2$, $3$, and   $4$ and  switch   to  the {\GAP}  window.   Use
`Selected' to get the subgroups associated with these vertices.

\begintt
gap> u := Selected( s );
[ Subgroup( c2, [ a, b*a^2*b^-2, b*a*b^2*a^-1*b^-1*a^-1*b^-1,
                  b^4*a^-2*b^-2, b^2*a^3*b^-1*a^-1*b^-2 ] ), 
Subgroup( c2, [ a, b^2*a*b^-1*a^-1*b^-1, b^3*a^-1*b^-1,
                b*a*b*a^3*b^-1 ] ), 
Subgroup( c2, [ a, b^2*a*b^-1*a^-1*b^-1, b*a^3*b^-2, b^4*a^-1*b^-3, 
                b*a*b^3*a^-1*b^-1 ] ) ] 
\endtt

`OperationCosetsFpGroup' computes for each of  these subgroups $u_i$  the
operation of  $C_2$ on its cosets. The  operation on $u_i$ is therefore a
permutation representation of the factor group $$C_2 / Core(u_i).$$ Using
`DisplayCompositionSeries' we can identify these factor groups.

\begintt
gap> p := List( u, x -> OperationCosetsFpGroup( c2, x ) );;
gap> for x  in p  do DisplayCompositionSeries(x);  Print("\n\n");  od;
<G> (2 gens, size 95040)
 I M(12)
<1> (0 gens)
 (2 gens, size 660)
 I A(1,11) = L(2,11) ~ B(1,11) = O(3,11) ~ C(1,11) = S(2,11)
 I ~ 2A(1,11) = U(2,11)
<1> (0 gens)
 (2 gens, size 239500800)
 I A(12)
<1> (0 gens) 
\endtt

So $C_2$  contains the Mathieu group  $M_{12}$, the alternating  group on
$12$ symbols and $PSL(2,11)$ as  factor groups.  Therefore it would  have
been possible to   find vertex $3$ using  <Epimorphisms>  instead of <Low
Index Subgroups>.  Close  the graphic sheet by  selecting  the menu entry
<close graphic sheet> from the <GAP> menu and start with a fresh one.

\begintt
s := GraphicSubgroupLattice(c2);
<interactive graphic lattice> 
\endtt

Select <Epimorphisms> from  the <Subgroups> menu.   This  pops up a  menu
similar to the <Information> menu  (see "GraphicSubgroupLattice for FpGroups,
Subgroups Menu").

\begintt
Sym(n)
Alt(n)
Sym(n) > Alt(n)
PSL(d,q)
Library
User Defined 
\endtt

Select  <Library>, which  pops   up a file  selector.   Choose  the  file
\begintt
l2_11.grp
\endtt and click on <OK>.  After a while the entry will change to

\begintt
Library (l2_11)      1 found 
\endtt

telling  you,  that  {\GAP}  has   found $1$  epimorphism   (up to  inner
automorphisms  of  $PSL(2,11)$) from  $C_2$   onto $PSL(2,11)$.  Click on
<display> to create a  new vertex representing a  subgroup $u$ such  that
the   factor group of  $C_2  /  Core(u)$  is  isomorphic to  $PSL(2,11)$.
Instead of <Library> you could have chosen <PSL(d,q)>.

This  is now  the  end of  the investigation   of the (partial)  subgroup
lattice  of $C_2$, you have  seen  that $C_2$   is infinite and  contains
$M_{12}$, $Alt(12)$, and $PSL(2,11)$ as factor groups.  Close the graphic
sheet by selecting <close graphic sheet> from the <GAP> menu.

%%%%%%%%%%%%%%%%%%%%%%%%%%%%%%%%%%%%%%%%%%%%%%%%%%%%%%%%%%%%%%%%%%%%%%%%%
\Section{A Partial Subgroup Lattice of the Trefoil Knot Group}

This  section  investigates the  following  finitely presented group, the
trefoil knot group $K_3$.
$$
    \langle a, b \;;\; aba = bab \rangle
$$

\begintt
gap> f := FreeGroup( "a", "b" );
Group( a, b )
gap> k3 := f / [ f.1*f.2*f.1 / (f.2*f.1*f.2) ];
Group( a, b )
gap> s := GraphicSubgroupLattice(k3);
<interactive graphic lattice> 
\endtt

If you compute the   abelian invariants of  $K_3$  you will see  that the
commutator factor group  is isomorphic to the  infinite cyclic group.  If
you try  to compute the   derived subgroups  {\GAP}  will print  a  short
warning

\begintt
#W  abelian quotient of 1 is infinite. 
\endtt

because  `GraphicSubgroupLattice'  cannot   deal  with subgroups of  infinite
index.  So  use <Prime Quotient> to   compute a $2$-quotient  of $K_3$ of
class $4$ instead.  This creates a new vertices $2$, $3$, $4$ and $5$ for
the $2$-quotients of  class $1$ up to $4$  which however have a different
shape. They are diamonds inside a square.  A diamond again indicates that
the associated subgroup  is normal.  A square is  meant as a danger sign,
whenever you see a square, {\GAP} has not yet  computed a coset table for
this subgroup in the group.  Because no coset  table is known, {\GAP} has
*not  checked* if this  subgroup is  equal or conjugate   to any of these
other  subgroups nor has  it checked  any  inclusions.  So it is possible
that  there are two  vertices on  the sheet (where    at least one has  a
square) which   represent  the *same* subgroup.    We   will see  such  a
constellation now.

In order to further investigate $K_3$, compute the  subgroups of index at
most $4$.  There is one normal subgroup of  index $2$ belonging to vertex
$8$ and one of index $4$.  However, because the quotient  of $K_3$ by the
subgroup of vertex $2$ is  the largest elementary abelian $2$-quotient of
$K_3$, the subgroups belonging to vertex  $2$ and $8$ must  be equal.  In
order  to check this,  pop up the  <Information> menu for vertex $2$, and
compute a coset table.  As soon  as {\GAP} knows  a coset table it checks
this table against the other known coset tables. If it discovers that two
subgroups are equal, {\GAP} deletes   one vertex and updates the  partial
lattice.  In this example, {\GAP} deletes vertex $8$, and connects vertex
$9$ with $2$.

You can force this check for a set of vertices by selecting the vertices,
in our example  $3$, $4$ and $5$, and  then <Check Coset Tables> from the
<CleanUp> menu.  But keep in mind that  especially the results of a prime
quotient can have huge indices.

This is  now   the end  of the investigation   of  the (partial) subgroup
lattice of $K_3$,  close the graphic  sheet  by selecting <close  graphic
sheet> from the <GAP> menu.

%%%%%%%%%%%%%%%%%%%%%%%%%%%%%%%%%%%%%%%%%%%%%%%%%%%%%%%%%%%%%%%%%%%%%%%%%
\Section{GraphicSubgroupLattice}

`GraphicSubgroupLattice( <g> )'

`GraphicSubgroupLattice' creates a new graphic sheet containing the Hasse
diagram of the subgroup lattice of <g>.  The next sections describe
how to select and move vertices and the following sections describe the
available menus.

\begintt
gap> GraphicSubgroupLattice( DihedralGroup(8) );
<graphic subgroup lattice "GraphicSubgroupLattice">
\endtt

`GraphicSubgroupLattice( <g>, <width>, <height> )'

In this form `GraphicSubgroupLattice' creates a graphic sheet of initial
dimensions <width> times <height>.  However it is still possible to change
these dimensions later using either "GraphicSubgroupLattice, Poset Menu".

This function is equal for all types of groups. It only behaves differently
according to some properties of the group. For example finitely presented
groups are treated differently because other algorithms apply in this case
and some of the standard ones are not feasible to carry out. 

In contrast to the {\GAP3} version of {\XGAP} all graphic subgroup lattices 
are interactive so you can always remove vertices. You can still get the
full lattice by choosing ``All Subgroups'' for the whole group.

%`GraphicSubgroupLattice( <g>, \"prime\"\ )'
%
%If you supply the parameter \"prime\", then a vertex will be placed above
%another  vertex  if the   number  of prime factors   in  the size  of the
%corresponding subgroup is larger than the number of  prime factors in the
%size  of  the  subgroup belonging   to  the second  vertex.   The default
%behaviour is to compare the sizes instead of number of prime factors.

%\begintt
%gap> GraphicSubgroupLattice( CyclicGroup(20) );;
%gap> GraphicSubgroupLattice( CyclicGroup(20), "prime" );; 
%\endtt

%`GraphicSubgroupLattice( <g>, \"normal subgroups\"\ )'

%If you supply the parameter  \"normal subgroups\", then only the  lattice
%of normal subgroups of <g> will be drawn on the graphic sheet.

%\begintt
%gap> GraphicSubgroupLattice( DihedralGroup(8), "normal subgroups" );
%\endtt

%It is possible to abbreviate \"prime\"\  and \"normal subgroups\"\ and to
%supply one or both.

%%%%%%%%%%%%%%%%%%%%%%%%%%%%%%%%%%%%%%%%%%%%%%%%%%%%%%%%%%%%%%%%%%%%%%%%%
\Section{GraphicSubgroupLattice, Moving Vertices}

In order to move a vertex, place the  pointer inside the vertex using the
mouse, and press the *left* mouse button.  Hold the mouse button pressed,
and start moving  the pointer by moving  the mouse.  The  vertex will now
follow the pointer, but it is not possible to move the vertex higher than
a vertex  of bigger size  
%(or number of prime  factors, if \"prime\"\ was
%given as parameter to `GraphicSubgroupLattice') 
or lower than a vertex of smaller
size.
%(or number of prime factors).  
As soon as you release the left mouse
button the vertex will stop following the pointer  and, if the vertex was
a member of a  conjugacy class, the remaining elements  of the class  are
moved.

If you hold  down  the <SHIFT> key before  moving  a vertex as  described
above then  only the selected vertex is moved but not all members of the
same class.

%%%%%%%%%%%%%%%%%%%%%%%%%%%%%%%%%%%%%%%%%%%%%%%%%%%%%%%%%%%%%%%%%%%%%%%%%
\Section{GraphicSubgroupLattice, Selecting Vertices}

Selected vertices are  represented by a slightly  thicker circle and,  if
your screen  supports color, are colored red   or green.  There  are five
different ways to select or deselect a vertex or a bunch of vertices.

Place the  pointer inside  a  vertex and press   the *left* mouse button.
Release   the button immediately   without  moving the  mouse.  This will
deselect all other vertices  and select this vertex.   If the vertex  was
already the only selected vertex it is deselected.

Place the  pointer inside a vertex, hold  down the <SHIFT> key, and press
the  *left* mouse button.   Release the button immediately without moving
the    mouse, release the   <SHIFT> key  afterwards.   If  the vertex was
deselected this action will  select it in  addition to any other selected
vertices. If the vertex was already selected it will be deselected.

Place the pointer  outside any vertex  and press the *left* mouse button.
Keep the button pressed and start moving the pointer.  This will create a
rubber band rectangle.  One corner at the position  where you pressed the
mouse button, the opposite corner following  the pointer.  As soon as you
release the left mouse button,   all  vertices inside the rectangle   are
selected, all vertices outside the rectangle are deselected.

Place the pointer outside any vertex, hold down the <SHIFT> key and press
the *left*  mouse button.  Again  you see a rubber  band, however, now as
soon as you release  the mouse button,  all vertices inside the rectangle
are selected in addition to any other selected vertices.

Call the function `SelectGroups' with a subgroup or list of subgroups to
select.

%%%%%%%%%%%%%%%%%%%%%%%%%%%%%%%%%%%%%%%%%%%%%%%%%%%%%%%%%%%%%%%%%%%%%%%%%
\Section{GraphicSubgroupLattice, GAP Menu}

The ``GAP'' menu will  be pulled down if  you place the pointer inside  the
``GAP'' button  and press the left mouse  button.  Keep the button down and
choose an entry by moving the pointer on  top of this entry.  Release the
mouse button to select an entry.

\beginitems
`save as postscript' &
Selecting this entry pops up a file  selector.  If you  enter a file name
and click on <OK>, this will save a  description of the graphic sheet and
graphic objects on  the sheet as encapsulated postscript  output. This
output should be imported easily into other documents.

`close graphic sheet' &
This entry will close the current graphic sheet and the window containing
the sheet  and all  ``Information'' menus related  to  this sheet.  On some
window system or using  some window manager the window  itself has also a
close button or menu entry.  However,  using the window close method will
not close associated ``Information'' menus.
\enditems

%%%%%%%%%%%%%%%%%%%%%%%%%%%%%%%%%%%%%%%%%%%%%%%%%%%%%%%%%%%%%%%%%%%%%%%%%
\Section{GraphicSubgroupLattice, Poset Menu}

The ``Poset'' menu will be pulled down if you place the pointer inside the
``Poset'' button and  press the left mouse  button.  Keep  the button down
and choose an entry by moving the pointer on top of  this entry.  Release
the mouse button to select an entry.

This menu is a generic menu for any graphic poset and is not specific to
subgroup lattices. So you find here options for the handling of general
posets. 

\beginitems
`Redraw' &
  The whole lattice will be redrawn. Use this option if some manipulation
  has disturbed the window.
  
`Show Levels' & 
  If this menu entry is activated (after clicking it will have a small
  check sign at its right, clicking again deactivates it), there will be a
  little blue box under each level at the left edge of the graphic
  sheet. You can use these little boxes to change the height of a level by
  moving it up or down like a vertex. In cases where the order of the
  levels is not given by some ``external source'' like the index of the
  subgroups you can move levels by moving the corresponding blue box with
  pressed <SHIFT> button.
  
`Show Levelparameters' &
  This menu entry controls the display of the level parameters at the right 
  edge of the graphic sheet. Normally these are displayed but you can
  switch it off with this menu entry.

`Delete Vertices' &
  Selecting the entry will delete all selected vertices. All edges from or
  to one of these are also deleted. However, inclusion information is
  preserved. This means that new edges are created from all vertices which
  were maximal in the deleted one to all vertices in which the deleted
  vertex was maximal. So you get the Hasse diagram of the poset which is
  the restriction of the former one to the not selected vertices.

`Delete Edges' &
  This menu entry is normally not selectable because it would destroy the
  Hasse diagram.

`Magnify Lattice' & 
  Selecting the entry will multiply the dimensions of the graphic sheet by
  the square root of $2$ and enlarge the lattice accordingly.

`Shrink Lattice' &
  Selecting  the entry will divide the  dimensions  of the graphic sheet by 
  the square root of $2$ and shrink the lattice accordingly.

`Resize Lattice' &
Selecting this entry will pop up a dialog box asking for an x and y factor
separated by a comma.  You can enter integers or quotients If you enter
only one number this is used for x and y.  The graphic sheet is then
enlarged or shrinked and the lattice is resized accordingly.

`Resize Graphic Sheet' &
This  entry is similar to `Resize  Lattice' except that  only the graphic
sheet is changed, the lattice remains unchanged. The numbers you enter must 
be integers and mean pixel numbers.

`Change Labels' &
Selecting  this entry will pop  up a dialog  box for each selected vertex
asking for a new label.  Clicking on <CANCEL>  will cancel the relabeling
of the remaining vertices but will not reset the already changed labels.

`Average Y Positions' &
Selecting this entry will average the y coordinates of all vertices
belonging to the same level. For graphic subgroup lattices this means als
subgroups with the same index in the whole group.

`Average X Positions' &
Selecting this entry  will  average the x   coordinates  of two or   more
selected vertices.  This will only work if the corresponding subgroups do
not have the same size and is used to align certain vertices
vertically.

`Rearrange Classes' &
Selecting this entry will clean up all classes which contain a selected
vertex. You need this option if you have moved a vertex without its class
(holding down the <SHIFT> key). The vertices in a class are arranged one
next to the other horizontally without changing the order of the x
coordinates. So you can permute the vertices within a class carelessly and
then again get a nice picture with the new order by selecting this menu
entry.

`Use Black\&White' &
Switches to black and white in case of a color screen or back to colors.
\enditems

%%%%%%%%%%%%%%%%%%%%%%%%%%%%%%%%%%%%%%%%%%%%%%%%%%%%%%%%%%%%%%%%%%%%%%%%%
\Section{GraphicSubgroupLattice, Subgroups Menu}

The ``Subgroups'' menu will be pulled down  if you place the pointer inside
the ``Subgroups'' button and press the left mouse  button.  Keep the button
down and choose an entry   by moving the pointer  on  top of this  entry.
Release the mouse button to select an entry.

The result   of a computation from   any of the  following  entries is
colored green, if your screen supports color.  There will also be a short
information message in the {\GAP} window about the result.

In the following descriptions   we use ``vertices'' as abbreviation  for
``subgroups associated with vertices''.

The following descriptions do not apply to the case of finitely presented
groups. See "GraphicSubgroupLattice for FpGroups, Subgroups Menu" for this
case. 

\beginitems
`All Subgroups' &
computes and displays all subgroups of the selected vertices. Requires at
least one selected vertex. Use with care! This can cause huge computations!
See also "ref:LatticeSubgroups".

`Centralizers' &
computes  and displays the   centralizers of  the selected  vertices with
respect to the whole group.  Requires at  least one selected vertex.  See
also "ref:Centralizer".

`Centres' &
computes and displays  the centre of  the selected vertices.  Requires at
least one selected vertex. See also "ref:Centre".

`Closure' &
computes and    displays the common  closure  of  the  selected vertices.
Requires at least one selected vertices. See also "ref:ClosureGroup".

`Closures' &
computes  and  displays the pairwise  closures of  the selected vertices.
Requires at least two selected vertices. See also "ref:ClosureGroup".

`Commutator Subgroups' &
computes and displays  the pairwise commutator  subgroups of the selected
vertices.    Requires  at  least    two   selected  vertices. See    also
"ref:CommutatorSubgroup".

`Conjugate Subgroups' &
computes and displays the conjugacy  classes (with  respect to the  whole
group) of each  selected vertex.  Requires  at least one selected vertex.
See also "ref:ConjugacyClass".

`Cores' &
computes and displays the cores of  the selected vertices with respect to
the whole group.  Requires at least one selected vertex. See also "ref:Core".

`Derived Series' &
For   each  selected vertex  `Derived  Series'  computes and displays its
derived series.   Requires  at   least  one selected vertex.    See  also
"ref:DerivedSeriesOfGroup".

`Derived Subgroups' &
computes and  displays the  derived  subgroups of the selected  vertices.
Requires at least one selected vertex.  See also "ref:DerivedSubgroup".

`Fitting Subgroups' &
computes and   displays   the   Fitting   subgroups   of   the   selected
vertices.  Requires    at   least one      selected  vertex.   See   also
"ref:FittingSubgroup".

`Intermediate Subgroups' &
computes and displays all intermediate subgroups between two selected
groups. Requires exactly two selected vertices. See also
"ref:IntermediateSubgroups". 

`Intersection' &
computes and  displays the common intersection  of the selected vertices.
Requires at least one selected vertices.  See also "ref:Intersection".

`Intersections' &
computes and displays the pairwise intersection of the selected vertices.
Requires at least two selected vertices.  See also "ref:Intersection".

`Normalizers' &
computes   and displays  the normalizers   of  the selected vertices with
respect to the whole group.  Requires at least  one selected vertex.  See
also "ref:Normalizer".

`Normal Closures' &
For   each  selected vertex `Normal  Closure'  computes  and displays the
normal closure of this vertex with respect  to the whole group.  Requires
at least one selected vertex.  See also "ref:NormalClosure".

`Normal Subgroups' &
For   each selected vertex `Normal  Subgroups'  computes and displays the
normal subgroups of the subgroup associated  with this vertex.  These new
subgroups are not  necessarily  normal in the  whole  group.  Requires at
least one selected vertex.  See also "ref:NormalSubgroups".

`Sylow Subgroups' &
pops  up a dialog  box asking   for prime.   After entering  a  prime and
pressing  <return> or  clicking  <OK> it computes   and displays  a Sylow
subgroup  for   each selected vertex.  Requires   at   least one selected
vertex.  See also "ref:SylowSubgroup".
\enditems

%%%%%%%%%%%%%%%%%%%%%%%%%%%%%%%%%%%%%%%%%%%%%%%%%%%%%%%%%%%%%%%%%%%%%%%%%
\Section{GraphicSubgroupLattice, Information Menu}

Note that this section does not deal with the case of a finitely presented
group. See "GraphicSubgroupLattice for FpGroups, Information Menu" for this 
case.

Placing the pointer  inside a vertex (selected  or not)  and pressing the
*right* mouse button pops up the  `'Information'' menu.  Clicking on any of
the text  lines will compute  the corresponding  property of the subgroup
$u$ associated  with  this vertex.  Clicking   on <all> will  compute all
properties, clicking on <close> will close the ``Information'' menu.

\beginitems
`Size' &
computes and displays the size of $u$.  See also "ref:Size".

`Index' &
computes and displays the  index  of $u$ in   the whole group.  See  also
"ref:Index".

`IsAbelian' &

`IsCyclic' &

`IsNilpotent' &

`IsPerfect' &

`IsSimple' &

`IsSolvable' &
computes and displays the corresponding property of $u$.  See also
"ref:IsAbelian", "ref:IsCyclic", "ref:IsNilpotentGroup", "ref:IsPerfectGroup",
"ref:IsSimpleGroup", and "ref:IsSolvableGroup".

`IsCentral' &

`IsNormal' &
computes and displays the corresponding  property of $u$ with respect  to
the whole group.  See also "ref:IsCentral" and "ref:IsNormal".

`Isomorphism' &
computes and displays  the isomorphism type of  $u$.  This will only work
if the size of $u$ is small.  See "ref:IdGroup" for details.
\enditems

%%%%%%%%%%%%%%%%%%%%%%%%%%%%%%%%%%%%%%%%%%%%%%%%%%%%%%%%%%%%%%%%%%%%%%%%%
%\Section{GraphicSubgroupLattice, Sheet Menu}
%
%In order to adjust any margin place the pointer outside of any vertex and
%press the  *right* mouse  button,  keep the button  pressed.  The <Sheet>
%menu will pop up.
%
%`Left Margin'
%
%selecting <Left  Margin> will narrow the  graphic sheet, setting  the new
%left margin to the x coordinate of your mouse click.
%
%`Right Margin'
%
%selecting <Right  Margin> will narrow the graphic  sheet, setting the new
%right margin to the x coordinate of your mouse click.
%
%`Top Margin'
%
%selecting <Top Margin> will narrow the graphic sheet, setting the new top
%margin to the y coordinate of your mouse click.
%
%`Bottom Margin'
%
%selecting <Bottom Margin> will narrow the graphic  sheet, setting the new
%bottom margin to the y coordinate of your mouse click.
%


%%%%%%%%%%%%%%%%%%%%%%%%%%%%%%%%%%%%%%%%%%%%%%%%%%%%%%%%%%%%%%%%%%%%%%%%%%
%\Section{GraphicSubgroupLattice, Vertex Shapes}
%
%The following vertex shapes can appear in an interactive lattice.
%
%circle:\\
%    conjugation for the subgroup associated with  the vertex was checked.
%    If the subgroup is conjugate  to any other  subgroup of a vertex both
%    vertices are drawn closely together.
%
%square:\\
%    conjugation was not  checked  for the  subgroup associated with  this
%    vertex.

%%%%%%%%%%%%%%%%%%%%%%%%%%%%%%%%%%%%%%%%%%%%%%%%%%%%%%%%%%%%%%%%%%%%%%%%%
\Section{GraphicSubgroupLattice for FpGroups, Subgroups Menu}

The <Subgroups> menu will be pulled down  if you place the pointer inside
the <Subgroups> button and press the  left mouse button.  Keep the button
down and  choose an entry by  moving  the pointer on   top of this entry.
Release the mouse button to select an entry.

The result of  of  a computation  from any  of  the following entries  is
colored green, if  your screen supports color.   In most cases there will
also be short information message in the {\GAP} window about the result.

In  the following descriptions, we  use ``vertices'' as abbreviation for
``subgroups associated with vertices''.

\beginitems
`Abelian Prime Quotient' &
pops  up a dialog  box asking  for a  prime  $p$.  It  then computes  and
displays the largest elementary   abelian  $p$ quotient of  the  selected
vertex.  If no presentation for the  subgroup associated to the vertex is
known   a presentation is  first computed  using  a modified Todd-Coxeter
algorithm.     It then calls    `PrimeQuotient' to   compute  the largest
elementary abelian quotient.  Requires exactly one selected vertex.

`All Overgroups' &
computes and displays all over-groups  of the selected vertex.  It  first
computes the permutation action of the whole  group on the cosets of the
subgroup associated with  the selected vertex and  then searches  for all
block systems.  Requires exactly one selected vertex.

`Conjugacy Class' &
computes and  displays  the   conjugacy  class of  the   selected vertex.
Requires  exactly  one selected vertex.  

`Cores' &
computes and  displays the cores  of the selected vertices.   Requires at
least one selected vertex.

`Derived Subgroups' &
computes and displays the derived subgroups of the selected vertices.  It
will  only  display  those derived    subgroups  whose index is   finite.
Requires at least one selected vertex.

`Epimorphisms' &
pops up another menu. Requires exactly one selected vertex.

\begintt
Sym(n)
Alt(n)
Sym(n) > Alt(n)
PSL(d,q)
Library
User Defined 
\endtt

Click on any of these entries to try to find a quotient isomorphic to the
symmetric group (`Sym(n)'), the alternating group (`Alt(n)'),
the symmetric or the alternating group (`Sym(n) > Alt(n)'), the projective
special linear group (`PSL(d,q)'), a group in a library supplied with
{\XGAP} (this will pops up a file selector), or a user defined group stored
in the variable `PERM_GROUP'.  After supplying additional parameters, for
example, the degree of the symmetric group or the dimension and field of
$PSL$ using dialog boxes, the corresponding entry will change, for example
to something like

\begintt
Sym(3)        3 found
\endtt

After one or more quotients were found click <display> to display them.

`Intersections' &
computes and   displays   the pairwise   intersections  of  the  selected
vertices.  Requires at least two selected vertices.

`Low Index Subgroups' &
pops up a dialog box asking for index  limit $k$.  It  will then do a low
index subgroup search for  subgroup of index at  most $k$ of the selected
vertex using `LowIndexSubgroupsFpGroup'.    If no  presentation  for  the
subgroup   associated to  the vertex  is  known a  presentation  is first
computed using a  modified  Todd-Coxeter algorithm. Requires  exactly one
selected vertex.

`Normalizers' &
computes and displays the normalizers of the selected vertices.  Requires
at least one selected vertex.

`Prime Quotient' &
pops up a dialog box asking for a prime $p$ and another dialog box asking
for a class $c$.  It then  computes and displays the largest $p$-quotient
of class $c$ of the selected vertex.  If no presentation for the subgroup
associated to the vertex is known a  presentation is first computed using
a   modified  Todd-Coxeter algorithm.   It    then calls `PrimeQuotient'.
Requires exactly one selected vertex.
\enditems

%%%%%%%%%%%%%%%%%%%%%%%%%%%%%%%%%%%%%%%%%%%%%%%%%%%%%%%%%%%%%%%%%%%%%%%%%
%\Section{GraphicSubgroupLattice for FpGroups, CleanUp Menu}
%
%`Check Coset Table'
%
%will compute  coset tables  for the selected   vertices and  checks these
%coset  tables against the other known  tables.  If equal coset tables are
%discovered the corresponding vertices are  merged.  Requires at least one
%selected vertex.
%
%`Deselect All'
%
%deselects all vertices.  Requires at least one selected vertex.
%
%`Delete Vertex'
%
%deletes  a selected vertex.   *Handle with  care*, if  the vertex is  the
%start  point  of  a  prime quotient  or   a derived subgroup all  related
%vertices are deleted.  Requires at least one selected vertex.
%
%`Use Black\&White'
%
%switches to black and white in case of a color screen.
%

%%%%%%%%%%%%%%%%%%%%%%%%%%%%%%%%%%%%%%%%%%%%%%%%%%%%%%%%%%%%%%%%%%%%%%%%%
\Section{GraphicSubgroupLattice for FpGroups, Information Menu}

Placing the pointer  inside a vertex (selected  or not) and  pressing the
*right* mouse button pops up the ``Information''  menu.  Clicking on any of
the text  lines will compute the corresponding   property of the subgroup
$u$  associated  with this  vertex. Clicking  on <close>  will  close the
``Information'' menu.  Let $s$ be the graphic sheet.

\beginitems
`Index' &
displays  the index of  $u$ in the whole  group.

`IsNormal' &
checks  if $u$ is  normal  in the  whole group.

`Abelian Invariants' &
computes  and  displays  the  abelian invariants  of  $u$.

`Presentation' &
computes a presentation of $u$ and displays  the number of generators and
relators   of   this presentation.

`Coset Table' &
computes   a   coset table for   $u$.

`Factor Fp Group' &
computes a presentation  for the factor group  of the whole group by $u$,
if $u$ is normal.


%%%%%%%%%%%%%%%%%%%%%%%%%%%%%%%%%%%%%%%%%%%%%%%%%%%%%%%%%%%%%%%%%%%%%%%%%
\Section{GraphicSubgroupLattice for FpGroups, Vertex Shapes}

The following vertex shapes can appear in an interactive lattice.

circle:\\
    a vertex for which  a coset table for  associated subgroup is  known.
    These vertices are checked for equality among each other.

square:\\
    a vertex for which  no coset table for  associated subgroup is known.
    These vertices are not checked for equality among all vertices.

diamond:\\
    as  a `circle' but   the associated subgroup  is  normal in the whole
    group *and* {\GAP} knows this.

diamond with square:\\
    as a  `square'  but the associated subgroup  is  normal in  the whole
    group *and* {\GAP} knows this.


%%% Local Variables: 
%%% mode: latex
%%% TeX-master: "manual"
%%% End: 
