\Chapter{Graphic Posets}

This chapter describes the part of {\XGAP} that allows the user to
conveniently display posets graphically.

\Section{Introduction}

A mathematical poset is just a partial ordered set. To display posets
reasonably in a generic way we need additional structure. So for {\XGAP} a
poset comes in so called levels. At all times in the life of a graphic
poset there are only finitely many levels and they are totally ordered,
that is for two levels we can always say, which one is ``higher''. The
position within the graphic sheet reflects this ordering.

The levels are parametrized by ``level parameters'', which can be any
{\GAP} object but must be unique within a graphic poset. A level is always
accessed by its level parameter and *not* by its number!

The vertices in each level are grouped into classes. For example for 
graphic subgroup lattices vertices in the same class correspond to conjugate
subgroups, vertices in the same level have the same size or index in the
whole group. The classes within each level are parametrized by ``class
parameters'', which can be any {\GAP} object but must be unique within a
level. A class within a level is always accessed by its class parameter and 
*not* by its number!

The user must supply a *partial order* for all of his levels. The mechanism 
to achieve this is the operation `CompareLevels', which compares two level
parameters. The current *total order* of the levels is always a refinement
of the partial order. The user can permute levels, if that does not
contradict the partial order defined by `CompareLevels'.

A vertex in the poset that is ``contained in'' another vertex in the poset
order (we speak of ``inclusion'' like in the case of subgroup lattices)
must always be in a level that is lower on the screen, because there 
is only a connecting line representing the inclusion. This is achieved by
the fact, that inclusions of vertices are communicated to {\XGAP} just by
creating an ``edge'' between them. This means, that the vertex in the
``lower'' level lies in the vertex in the ``higher'' level. There must not
be edges between vertices in the same level!

The terminology ``vertices'' and ``edges'' comes from the fact, that a
graphic poset is just a special case of a graphic graph, where vertices can 
be placed anywhere in the sheet and edges have nothing to do with
inclusion. It is planned that also a graphic graph library is implemented
in {\XGAP} but it is not yet operational. However everything which could be 
done not only for posets but at the same time for graphs is implemented
already within the poset package. This explains the usage of ``graph'' in
many places where you would otherwise expect ``poset'.

What you have to do to use the graphic poset package is create a graphic
poset (a special instance of a graphic sheet), create some levels and
perhaps classes within them. Then you can create vertices and edges, to
encode the ordering. Everything else is done by the library. See the next
section for details about the available operations.

Note that we chose a functional approach for certain decision
procedures. This means that if you for example create a vertex and do not
specify a position, an operation (`ChoosePosition') is called to determine
the actual position. You can use the generic routines or install your own
methods for all of those decisions. In this case you just set a new filter
for your posets and overload the generic methods by special routines for
objects with your new filter set. You can see this approach in the example
in "An Example".

\Section{Operations}

\Input{posetops}

\Section{An Example}

This section shows how to use the poset package to display posets. The code
presented here is actually part of the {\XGAP} library and makes up the
link to the C MeatAxe.

This is the declaration part:

\beginexample
#############################################################################
##
#W  meataxe.gd                  XGAP library                  Max Neunhoeffer
##
#H  @(#)$Id: posets.tex,v 1.2 1999/01/17 23:44:15 gap Exp $
##
#Y  Copyright 1998,       Max Neunhoeffer,              Aachen,       Germany
##
##  This file contains declarations for meataxe posets
##
Revision.pkg_xgap_lib_meataxe_gd :=
    "@(#)$Id: posets.tex,v 1.2 1999/01/17 23:44:15 gap Exp $";

DeclareFilter("IsMeatAxeLattice");

#############################################################################
##
#O  GraphicMeatAxeLattice(<name>, <width>, <height>)  . creates graphic poset
##
##  creates a new graphic meataxe lattice which is a specialization of a
##  graphic poset. Those posets have a new filter for method selection.
##
DeclareOperation("GraphicMeatAxeLattice",[IsString, IsInt, IsInt]);
\endexample

The code only declares a new filter and declares a constructor operation
for posets that lie in this new filter.

The implementation:

\beginexample
#############################################################################
##
#W  meataxe.gi                  XGAP library                  Max Neunhoeffer
##
#H  @(#)$Id: posets.tex,v 1.2 1999/01/17 23:44:15 gap Exp $
##
#Y  Copyright 1998,       Max Neunhoeffer,              Aachen,       Germany
##
##  This file contains code for meataxe posets
##
Revision.pkg_xgap_lib_meataxe_gi :=
    "@(#)$Id: posets.tex,v 1.2 1999/01/17 23:44:15 gap Exp $";

#############################################################################
##
#M  GraphicMeatAxeLattice(<name>, <width>, <height>)  . creates graphic poset
##
##  creates a new graphic meataxe lattice which is a specialization of a
##  graphic poset. Those posets have a new filter for method selection.
##
InstallMethod( GraphicMeatAxeLattice,
    "for a string, and two integers",
    true,
    [ IsString,
      IsInt,
      IsInt ],
    0,

function( name, width, height )
  local P;

  P := GraphicPoset(name,width,height);
  SetFilterObj(P,IsMeatAxeLattice);
  return P;
end);

#############################################################################
##
#M  CompareLevels(<poset>,<levelp1>,<levelp2>) . . . compares two levelparams
##
##  Compare two levelparams. -1 means that levelp1 is "higher", 1 means
##  that levelp2 is "higher", 0 means that they are equal. fail means that
##  they are not comparable. This method is for the case if level
##  parameters are integers and lower values mean lower levels like in the
##  case of meataxe lattices of Michael Ringe.
##
InstallMethod( CompareLevels,
    "for a graphic meataxe lattice, and two integers",
    true,
    [ IsGraphicPosetRep and IsMeatAxeLattice, IsInt, IsInt ],
    0,
function( poset, l1, l2 )
  if l1 < l2 then
    return 1;
  elif l1 > l2 then
    return -1;
  else
    return 0;
  fi;
end);
\endexample

Besides the new contructor (which only adds a new filter) we only have to
supply a new method for comparison of level parameters for such posets. The 
levels are numbered with integer numbers such that lower numbers are lower
in the lattice.

There is a C program in the MeatAxe that exports a poset to a {\GAP}
program which generates the lattice in a graphic poset sheet. The user can
then interactively move around vertices and shrink or magnify levels. He
can then export the resulting lattice to an encapsulated postscript file.

