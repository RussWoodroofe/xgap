\Chapter{Differences to XGAP 3}

This rather short chapter is intended for the user who knows {\XGAP} 3 well
and quickly wants to know what has changed. So it covers mainly those
parts, where existing code using {\XGAP} has to be changed. For the totally 
new features and packages there are only a few references to the other
parts of the documentation.

%%%%%%%%%%%%%%%%%%%%%%%%%%%%%%%%%%%%%%%%%%%%%%%%%%%%%%%%%%%%%%%%%%%%%%%%%%%%%
\Section{Concept}

There are two main changes in the concept. The first is the migration to
{\GAP4} with all the bells and whistles like object oriented design with
operations, methods and method selection via filters. {\XGAP4} is rewritten 
nearly totally with these technologies. This should make the reusage of
code in the future easier. One can now use big parts of the code of {\XGAP}
for own structures by just replacing some methods via overloading.

The second change is that there is no longer any mathematical ``knowledge''
or algorithm in {\XGAP}. It is now only a front end and a graphical user
interface. All code for finitely presented groups resides now in the {\GAP}
library. This is a much cleaner concept and should make the management of
the source code easier. At the same time {\XGAP} has become a much more
generic program. Operations for subgroups are for example no longer hard
wired into {\XGAP} case by case but there is generic code which can be
adapted just by hacking a few tables.

These generalizations made some sacrifices necessary, because {\XGAP} does
no longer know anything about the mathematics it is displaying. It may for
example happen that {\XGAP} does no longer adapt its behaviour to the
amount of data that is known about some finitely presented groups. The
reason for this is, that the generic poset routines cannot know that the
vertices stand for groups at all. So sometimes one has to trigger the
comparison of subgroups of finitely presented groups manually (see section
"Compare Subgroups" for a description how to do this).

In the old {\GAP3} version of {\XGAP} there were three different programs
for the full subgroup lattice of a (finite) group (`GraphicLattice'), the
interactive partial subgroup lattice of a finite group
(`InteractiveLattice') and the interactive partial subgroup lattice of a
finitely presented group (`InteractiveFpLattice') respectively. Now there
is only one generic program to display subgroup lattices interactively
(`GraphicSubgroupLattice').

{\XGAP} can now handle subgroups of infinite index. They are either
placed in a ``finite size'' level or in an ``infinity'' level. See
"levelsintro" for details.

A new logging facility allows to automatically produce a protocol of
the actions the user performs via mouse clicks. This is convenient
because the normal {\GAP} command script contains no useful
information about the selected entries in the menus. See
"loggingfacility" for details.

There is a new layer to display generic posets that do not have to be
subgroup lattices. It can be used to display posets interactively very
easily. This is for example used in the new link to the C-MeatAxe written
by Michael Ringe. The code for this link is also included in {\XGAP4}.

Code for the display of graphic graphs is planned but not yet completed.

The user of {\XGAP} should not realize much of those changes (except of
course the name of the function to display a subgroup lattice). The
programmer on the other hand has to get used to the new techniques. It was
not in all places possible to achieve total compatibility for existing
code. Some changes also were introduced deliberately to make the
programmers adapt their programs to the new situation!


%%%%%%%%%%%%%%%%%%%%%%%%%%%%%%%%%%%%%%%%%%%%%%%%%%%%%%%%%%%%%%%%%%%%%%%%%%%%%
\Section{User Interface}

Some menu entries have been moved to new places, mainly because of the
division of generic poset code and specialized code for graphic subgroup
lattices. There are some new features and nearly all old features have made 
it into the new version.

The handling of the mouse is unchanged. However the introduction of levels
gives the user new possibilities.

%%%%%%%%%%%%%%%%%%%%%%%%%%%%%%%%%%%%%%%%%%%%%%%%%%%%%%%%%%%%%%%%%%%%%%%%%%%%%
\Section{Where code has to be changed}

All {\GAP} objects corresponding to graphic sheets and graphic objects are
no longer records but component objects. This means that the programmer can 
no longer mess around in the data structures. If you want to add new
fields, then you have to use inheritance and define new categories. This
means also that the (internal) data structures of sheets has changed
massively. Programs that try to access record components of old {\XGAP}
structures will no longer work!

The operation `InstallGSMethod' is no longer present. It is replaced by the 
``callback'' mechanism with the operations `InstallCallback',
`RemoveCallback' and `Callback' (see "GraphicSheet" for details). This
means, that mouse events are handled differently. This was changed
deliberately because there is a big difference: In {\XGAP4} you can install 
more than one function for one type of mouse event. All such callback
functions are called one after the other. There was only one graphic sheet
method for each event in {\XGAP3}. So you can *not* just change the name of 
the operation to install the callback. You have to think about this
difference! 

See the section "Operations for Graphic Objects" for an overview
which operations exist now for which graphic objects. The main difference
is the introduction of `Revive', `ViewObj' and `WindowId' together with the 
concept of the `IsAlive' filter.

There was a bug in {\XGAP3} in the creation of menus: If an entry starts
with a minus sign, it will become a separating line instead of a real menu
entry. This disturbed the numbering of the menu entries, such that `Enable' 
and `Check' did not work on the correct entry. This bug is fixed in
{\XGAP4} so code which contained a workaround for this bug has to be
changed. `Enable' and `Check' behave now like expected and documented in
"Enable" and "Check".


%%% Local Variables: 
%%% mode: latex
%%% TeX-master: t
%%% End: 
