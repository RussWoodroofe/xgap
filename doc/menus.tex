\Chapter{User Communication}

{\XGAP} has two main means to communicate with the user. The first is the
normal command processing: The user types commands, they are transmitted to 
{\GAP}, are executed, and produce output, which is displayed in the command
window. The second is the mouse and the other parts of the graphical user
interface. This latter part can be divided into menus, mouse events,
dialogs, and popups.

\beginitems
Menus & Most of the windows of {\XGAP} have menus. The user can select
entries in them and this information is transformed to a function call in
{\GAP}. Menu entries can be checked or not, so menus can also display
information. 

Mouse Events & A mouse event is the pressing or releasing of a mouse
button, together with the position where the mouse pointer is in the exact
moment this happens and the state of certain keyboard keys (mainly shift
and control). Such events also trigger {\GAP} function calls and the
corresponding functions can react on these events and for example wait for
others. 

Dialogs & Dialogs are windows which display information and into which
the user can enter information for example in form of text fields.

Popups & Popups are special dialogs where the user can not type text but
can only click on certain buttons. {\XGAP} has so called ``text selectors'' 
which are a convenient contruct to display textual information and let the
user select parts of it.
\enditems

Most of those graphical objects have corresponding {\GAP} objects, which
are created by contructors and can be used later on by operations.

\Input{usercomm}

